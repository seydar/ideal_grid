\documentclass{article}
\usepackage{hyperref}

\title{Electric Grid Modeling}
\author{Ari Brown}
\date{\today}

\begin{document}
\maketitle

\section{Introduction}
In my personal time, I have been developing a model of the US electric grid that
allows for modeling all transmission lines, generators, and substation loads at
the level of the RTO/ISO. I currently only have the data for New England
(ISO-NE), but I am hoping to get data for other regions as well.

\section{Heilmeier Questions}
\begin{enumerate}
  \item
  \textbf{What are you trying to do? Articulate your objectives using absolutely
  no jargon.} \\

  I am trying to do four things:
  \begin{enumerate}
    \item Deduce the best location to build new electrical generators
    \item Deduce the transmission lines that have the most electricity flowing
    through them
    \item Deduce the best location to build new transmission lines to reduce the
    load borne by high-flow transmission lines
    \item Mathematically define the resilience of an area and assess a given
    area for its resiliency
  \end{enumerate}

  \item
  \textbf{How is it done today, and what are the limits of current practice?} \\
  Sandia National Laboraties (SNL) has built their Resilience Analysis Process
  (RAP), which is a qualitative assessment of a given area. The SNL evaluation
  does not appear to use graph theory to determine resilience. Their process is
  labor-intensive and restricted to only those who work with SNL. The SNL RAP is
  about the structural integrity of individual elements, as opposed to an
  assumption that key lines can and will fail.

  Breakthrough Energy offers the ability to model the US at a very high level
  (whole country, Eastern Interconnection, Western Interconnection, and Texas).
  As a consequence of their high level, they do not have the ability to model
  regional-level details. In addition, their solution to grid congestion is to
  simply upgrade those congested lines, as opposed to building new lines
  elsewhere.

  \item
  \textbf{What is new in your approach and why do you think it will be
  successful?} \\
  My approach is to use graph theory to determine the resilience of an area
  based on excess capacity in the lines and a calculation of inflowing and
  outflowing lines to an area. I think it will be successful because it will
  allow for easy analysis of smaller areas, as well as a graph-theory-based
  approached to identify the best ways to improve reliability.

  \item
  \textbf{Who cares? If you are successful, what difference will it make?} \\
  There are two discussions surrounding the construction of transmission lines:
  grid resiliency and grid congestion.

  If I am successful, I will offer an analysis of grid resiliency that the
  Department of Energy and the Department of Defense does not (as far as I know)
  appear to have. This will encourage and enable DoE's and DoD's national
  security discussions as they push for government funding to develop new
  transmission lines.

  In addition, if I am successful, my analyses of grid congestion will enable
  small energy generation companies and Think Tanks to shift their demands
  away from building bigger transmission lines to handle more congestion and
  towards small lines in order to reduce congestion.

  The major benefit will be to convince the government to provide financial
  subsidies for construction of certain lines based on the analyses from my
  model.

  \item
  \textbf{What are the risks?} \\
  I do not have access to the 15-minute generator data, which would tell me
  which generators are producing how much energy -- this is crucial for
  determining the impact of congestion.

  The model currently does not support having a single load partially supported
  by multiple generators. I do not know how big of an impact this has, given
  other facets of the flow calculation algorithm.

  I do not have the political in-roads built to find costumers who could use
  this.

  Many of the battles are political; I need to map political land ownership in
  order to assess feasibility of new construction. Alternatively, my model can
  offer 10 suggestions, and a human can then independently assess those.

  I do not have access to a map of the fully connected grid. All of the data
  available appears to be limited in some capacity, which means my analyses
  would have reduced accuracy in the prediction of the effects of new
  construction.

  \item
  \textbf{How much will it cost?} \\
  More work needs to be done on the model to make it usable, and more work will
  be required in order to meet specific customer needs. The only costs are
  personnel.

  \item
  \textbf{How long will it take?} \\
  I need 3 months of full-time work to make the model initially usable to
  customers and to reduce the computation costs involved.

  \item
  \textbf{What are the mid-term and final ``exams'' to check for success?} \\
  Mid-term ``exams'' will be modeling individual RTOs/ISOs and checking that the
  transmission loss predicted by the model matches the transmission loss
  reported by the Energy Information Administration. The final ``exam'' would be
  testing that the level of the entire US.

\end{enumerate}

\end{document}

